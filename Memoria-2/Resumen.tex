\chapter*{Resumen}
\hspace{1cm} Es cada vez mas comun el uso de drones en el dia a dia de las personas. Se puede observar como han explotado en estos ultimos años como un aparato de entretenimiento. Pero estos tienen mucha historia, los UAV se comenzaron a utilizar hace muchos años con fines belicos, y poco a poco su desarrollo ha permitido que se puedan utilizar en ambitos muy diferentes, como puede ser en el campo de la rob\'otica. \\

\hspace{1cm} Durante este proyecto se ha desarrollado un algoritmo con la finalidad de que el dron navegue de forma autonoma y no necesite a ninguna persona diciendole lo que tiene que hacer en cada momento o controlandole, sino que se pueda confiar en el una vez abandone el punto de origen. \\

\hspace{1cm} Para conseguir esto principalmente se ha trabajado en la detecci\'on visual objetos, haciendo esto a partir de filtros de color, que nos permite detectar si un objeto es o no el deseado y de esta forma poder dirigirnos a el. Cabe destacar que este tiene que ser un objeto bien elegido y que sea dificil de confundir con los demas, para poder asegurarnos de un correcto funcionamiento. \\

\hspace{1cm} Tambi\'en se ha estudiado la forma de funcionamiento del drone con el que se va a trabajar, sus modos de comunicaci\'on y su din\'amica de vuelo, as\'i como las posibilidades que tiene este, hasta donde puede llegar y lo que se podr\'ia mejorar utilizando otros drones y elementos diferentes. \\

\hspace{1cm} Hay que destacar los distintos softwares utilizados, los cuales nos permiten un desarrollo, permitiendo tener un algoritmo fluido y dinamico. 

 

