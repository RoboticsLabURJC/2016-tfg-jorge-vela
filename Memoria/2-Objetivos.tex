\chapter{Objetivos}\label{cap.Objetivos}
En este cap\'itulo se van a explicar los objetivos a conseguir y el m\'etodo utilizado para llegar a ello.

\section{Objetivo principal}
\hspace{1 cm} El objetivo propuesto para este trabajo era conseguir que un drone navegara de forma aut\'onoma. Para ello, la idea era situar el drone en un punto de inicio, despegar sobre \'este y una vez se hubiera estabilizado comenzara a desplazarse siguiendo un algoritmo de b\'usqueda previamente programado. La idea es recorrer determinado \'area sin la necesidad de pilotar el drone. Los movimientos del drone depender\'ian de lo que detectara la c\'amara en cada momento. Esta b\'usqueda se realizar\'ia con la finalidad de encontrar una baliza sobre la cual aterrizar. En caso de no ver la baliza, el drone se mover\'ia realizando una espiral en b\'usqueda de \'esta y en el caso de encontrarla, el drone se situar\'ia sobre ella y, una vez centrado, aterrizar\'ia. Esta baliza ser\'ia un cuadrado formado por cuatro cuadrados de colores distintos. 

\begin{figure}[H]
	\centering
		\includegraphics{imgs/baliza.jpg}
	\label{fig:Baliza elegida sobre la que aterrizar. }
\end{figure}

\hspace{1 cm} Para conseguir este objetivo, hay que tener en cuenta varias partes importantes, de las cuales a partir de una pasamos a la otra para al final llegar a lo que permitir\'a el correcto funcionamiento del algoritmo: que los movimientos de este dependan de lo que ve la c\'amara.

\begin{itemize}
	\item \textbf{Creaci\'on de un filtro de color:} Gracias a los colores podemos detectar los cuadrados de la baliza, por lo que realizando el filtro obtendremos estos objetos y los que sean del mismo color, eliminando as\'i informaci\'on no necesaria de la imagen.
	
	\item \textbf{Detecci\'on de objetos: } Tras el filtro de color y eliminada informaci\'on no necesaria, podemos obtener distintos objetos en la imagen, viendo si cumplen caracter\'isticas como tener un determinado area, para as\'i saber si son los objetos que se desean o no y trabajar con ellos o descartarlos. 
	
	\item \textbf{Movimiento del drone: } Una vez obtenemos informaci\'on de la imagen y la hemos procesado, en funci\'on de esto se pretender\'a que el desplazamiento del drone sea uno u otro, trabajando as\'i de forma fluida y sin movimientos bruscos que puedan desestabilizarlo.   

\end{itemize}


\section{Metodolog\'ia}
\hspace{1 cm} Para poder conseguir todo esto, era necesario saber las plataformas con las que ser\'ia posible, donde tiene gran importancia el software JdeRobot, pues gracias a su desarrollo que permite la comunicaci\'on con el dron, y de esta forma obtener los datos de sus sensores, principalmente de la c\'amara. Para este apartado fue primordial el aprendizaje de sus distintas herramientas. Hubo que hacer pruebas con estas herramientas y ver su funcionamiento tanto en entornos simulados como en reales.

\hspace{1 cm}Una vez se realizaron las primeras pruebas con robots, se propuso un desarrollo en espiral. Para este desarrollo se propon\'ian semanalmente reuniones con el tutor, en las cuales se presentaban unos objetivos a seguir en funci\'on de lo que se hab\'ia conseguido hasta el momento. Una vez propuestas estas tareas se evaluaban los distintos riesgos que se pod\'ian tomar en funci\'on de trabajar de una forma u otra, y los avances a los que se pod\'ia llegar por cada camino. Una vez hecho esto, se comenzaban a desarrollar los algoritmos para conseguir estos objetivos en funci\'on de la forma elegida. Despu\'es se probaban (en simulaci\'on o con el robot real, dependiendo del objetivo), se propon\'ia otra reuni\'on para determinar si los avances eran los deseados o no, y en funci\'on de esto planear unos objetivos u otros.  

\begin{figure}[H]
	\centering
		\includegraphics{imgs/metodologia-espiral.jpg}
	\label{fig:Desarrollo en espiral}
\end{figure}

\hspace{1 cm} Los avances que se iban realizando semana a semana, se sub\'ian mediante v\'ideos o im\'agenes a un mediawiki en JdeRobot, quedando reflejados todos los procesos. Tambi\'en se ha utilizado GitHub para subir el c\'odigo desarrollado.  

\section{Plan de trabajo}
\hspace{1 cm} El primer punto fue comprender las distintas herramientas de JdeRobot y trabajar con ellas, creando programas simples y viendo que funcionaban, as\'i como trabajar con distintos robots reales para tener una toma de contacto con ellos y no basarse solo en un trabajo sobre simulador. 

\hspace{1 cm} En segundo lugar hubo que comprender la importancia que tendr\'ia la biblioteca OpenCV en este proyecto y ver sus m\'ultiples opciones, todo lo que permit\'ia trabajar con im\'agenes y los cambios que pod\'iamos hacer para obtener datos deseados a partir de los cuales mandar informaci\'on. 

\hspace{1 cm} Por \'ultimo quedaba unir estas dos tecnolog\'ias y conseguir un buen algoritmo que nos permitiera enviar informaci\'on a tiempo real a un drone y que \'este la ejecutara de forma correcta y fluida.  




