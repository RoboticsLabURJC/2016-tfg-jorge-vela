\chapter{Conclusi\'ones}\label{cap.conclusiones}

\hspace{1cm} Para finalizar, hay que realizar una evaluaci\'on del trabajo realizado, lo aprendido durante este periodo y los objetivos cumplidos. En un primer aspecto, podemos decir que el trabajo ha sido satisfactorio, se ha conseguido un algoritmo que desarrolle un despegue-b\'usqueda-aterriza como se deseaba en un primer momento, adem\'as que para llegar a ello se han superado distintos retos que han ido surgiendo a lo largo de su desarrollo.

\begin{itemize}
	\item El primera parte era realizar un filtro de color para poder aislar los objetos interesantes del resto. Este objetivo se ha conseguido, trabajando principalmente con la librer\'ia OpenCV. Destacar que en el simulador esto fue m\'as sencillo debido a la nitidez de los colores y que en el drone real llev\'o mas trabajo. Adem\'as se han tenido que hacer diversas pruebas debido a los distintos lugares donde se probaba el drone y la diferencia de colores que hab\'ia debido a la luminosidad, pero al final se consigui\'o un buen filtro, que con ayuda de los operadores morfol\'ogicos (erosi\'on, dilataci\'on, cierre y apertura), nos permit\'ian obtener una imagen de fondo negro y los objetos de los colores deseados en primer plano.

	\item La segunda parte trataba de, una vez obten\'iamos los objetos de los colores deseados, ver si estos eran los objetos que quer\'iamos o no. Por una parte, se miraba el \'area que ten\'ian las figuras, y en caso de no llegar a un tamaño predeterminado, estas se descartaban como posibles objetos. Por otro lado, nos apoyamos en la forma de la baliza, pues los 4 cuadrados que la compon\'ian formaban una cruceta en su centro, la cual era la que se trataba de detectar, por lo que no se depend\'ia s\'olo de unos colores determinados, sino tambi\'en de una figura. 
	
	\item Ya por \'ultimo, hablando de la parte de control del drone, decir que en el drone real ha sido mucho m\'as costosa de lo esperado, es una parte que sobre el simulador ha funcionado sin problemas cada vez que se añad\'ia algo, pero al querer trabajar sobre el dron real era muy complicado y cualquier cambio de un valor pequeño supon\'ia gran cambio en la realidad. Todo esto tambi\'en ha sido importante para ver la diferenc\'ia que hay cuando tienes s\'olo un control proporcional y despu\'es le añades las componentes integral y derivativa. 
	
\end{itemize}

\hspace{1cm} Las distintas pruebas y los avances que se han ido dando durante el desarrollo, se han ido validando y est\'an disponibles en la wiki oficial del proyecto:\\
\underline{\url{http://jderobot.org/Jvela-tfg}}

\hspace{1cm} Por otro lado, mirando los proyectos ateriores a este en los que hab\'ia trabajado RoboticsLabs URJC con drones, podemos ver que se ha conseguido algo diferente, ya que en este caso hemos conseguido que un drone navege de forma aut\'onoma gui\'andose por las balizas de color, consiguiendo que vaya de un punto a otro. 


\section{Lineas futuras}

\hspace{1cm}Es importante destacar que estos campo de la rob\'otica y la visi\'on se esta produciendo un importante crecimiento, y añadiendo lo aprendido en el trabajo y como se puede trabajar en \'el creo que ser\'ia interesante continuar la l\'inea de este proyecto para continuar con el desarrollo y aprendizaje de estas t\'ecnicas. Por un lado, ya hab\'iendo trabajado en la autolocalizaci\'on mediante balizas de colores, ser\'ia interesante trabajar en eso mediante otros sensores como puede ser el GPS, ya que es algo que se est\'a utilizando mucho actualmente y tiene grandes ventajas. Por otro lado, habiendo trabajado con el software de JdeRobot, ser\'ia interesante seguir viendo las opciones que tiene \'este y trabajar en sus herramientas para segu\'ir sacando provecho a los drones que cada vez son mas completos y vienen con m\'as sensores, por lo tanto tienen un mayor \'area de trabajo. 





