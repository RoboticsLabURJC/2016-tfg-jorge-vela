
\chapter{Conclusi\'ones}\label{cap.conclusiones}
\hspace{1cm} Para finalizar, hay que realizar una evaluaci\'on del trabajo realizado, lo aprendido durante este periodo, los objetivos cumplidos y a los que no hemos llegado de la forma deseada. En un primer aspecto, podemos decir que el trabajo ha sido satisfactorio, en primer lugar porque se ha conseguido un algoritmo que desarrolle un despegue-busqueda-aterriza como se deseaba en un primer momento, ademas que para llegar a ello se han superado distintos retos que han ido surgiendo a lo largo de su desarrollo, pero profundizando un poco mas en todo esto:
\begin{itemize}
	\item En primer lugar comentar todo lo que ha supuesto trabajar con un dron. En realidad creo que esto me ha aportado bastante para trabajos futuros, pues el trabajar con un robot real te lleva a darte cuenta de como puede cambiar algo en funci\'on de factores externos, como pueden ser viento, luz o lluvia, o incluso internos como el desgaste del robot, ya que no son todo unas condiciones ideales en las que superando ciertos tests sabes que el programa va a funcionar, sino que hay que hacer muchos test y en condiciones muy distintas. Ademas, tambi\'en cuando se da un inesperado problema y tienes que dejar de trabajar en la linea que lo estabas haciendo durante un tiempo determinado, uno se da cuenta que el trabajar en la realidad y no con simuladores puede conllevar esto, y tener que cambiar durante este tiempo la linea de trabajo sin tener claro como se van a juntar luego ambos caminos. Todo esto creo que son factores negativos que aportan datos positivos que ayudan a madurar en la forma de trabajo.
	
	\item En segundo lugar, entrando en el objetivo, yo creo, mas complicado de este trabajo, es el conseguir un filtro de color. Al final se ha conseguido un filtro deseado, pero tras muchos cambios. Esta parte ha llevado un gran trabajo, pues he tenido que aprender a utilizar herramientas nuevas y consultar muchas dudas que iban surgiendo, trabajar con distintos espacios de color y en diferentes condiciones de luminosidad. Pero al final este objetivo se ha conseguido, algo que ha resultado de gran satisfacci\'on puesto que se empezo pr\'acticamente al principio y no ha dejado de sufrir cambios hasta el ultimo momento.
	
	\item Ya por ultimo, hablando de la parte de control del dron, decir que en el dron real ha sido mucho mas costosa de lo esperado, es una parte que sobre el simulador ha funcionado sin problemas cada vez que se añadia algo, pero al querer trabajar sobre el dron real era muy complicado y cualquier cambio de un valor pequeño suponia gran cambio en la realidad. Es este punto tambi\'en uno se da cuenta de lo dificil que es dejar de lado las condiciones ideales. Aun asi ver el algoritmo de busqueda y el aterrzaje cuando encontraba la baliza ha sido muy fruct\'ifero.
\end{itemize}

\hspace{1cm} Tambi\'en, viendo la wiki realizada durante el proyecto, en la que se han ido documentando las cosas importantes que se iban haciendo, y algunos datos que tenia guardados de diversas pruebas realizadas, puedo ver grandes avances en lo realizado, y darle un gran uso a las nuevas tecnolog\'ias.

\hspace{1cm} Es importante destacar que estos campo de la rob\'otica y la visi\'on se esta produciendo un importante crecimiento, y añadiendo lo aprendido en el trabajo y como se puede trabajar en el creo que ser\'ia interesante continuar la linea de este proyecto para continuar con el desarrollo y aprendizaje de estas t\'ecnicas. 



