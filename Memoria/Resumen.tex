\chapter*{Resumen}
\hspace{1cm} Es cada vez m\'as com\'un el uso de drones en el d\'ia a d\'ia de las personas. Se puede observar c\'omo han explotado en estos \'ultimos años como un aparato de entretenimiento. Pero los UAV tienen mucha historia,se comenzaron a utilizar hace muchos años con fines b\'elicos, y poco a poco su desarrollo ha permitido que se utilicen en \'ambitos muy diferentes, como la rob\'otica, gracias, por ejemplo, al comportamiento aut\'onomo de este.

\hspace{1cm} Durante este proyecto se ha diseñado y programado un algoritmo con la finalidad de que el drone navegue de forma aut\'onoma, desde una baliza inicial hasta otra baliza final, de posici\'on desconocida, y no necesite a ninguna persona dici\'endole lo que tiene que hacer en cada momento o control\'andole. Para conseguirlo se han tenido en cuenta los distintos estados en los que se puede encontrar el drone (despegando, b\'usqueda o aterrizaje).
Tambi\'en se ha desarrollado la percepci\'on visual de las balizas, utilizando filtros de color y otras t\'ecnicas, como operadores morfol\'ogicos, que permite detectar si un objeto es o no el deseado y de esta forma poder dirigirnos a el. Las balizas han sido elegidas para que sea dif\'icil de confundir con los dem\'as, y facilitar un correcto funcionamiento. Para toda esta programaci\'on nos hemos apoyado en el entorno JdeRobot, utilizando OpenCV para el procesado de im\'agenes.

\hspace{1cm} El algoritmo programado se ha validado experimentalmente, tanto en un drone simulado (utilizando el simulador Gazebo), como en un drone real (utilizando el Ardrone2 de Parrot). Se ha conseguido un comportamiento satisfactorio funcionando en tiempo real. Todo el software desarrollado y los v\'ideos de las pruebas est\'an accesibles p\'ublicamente. 


