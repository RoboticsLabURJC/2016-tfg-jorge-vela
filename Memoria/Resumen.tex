\chapter*{Resumen}
\hspace{1cm} Es cada vez m\'as com\'un el uso de drones en el d\'ia a d\'ia de las personas. Se puede observar c\'omo han explotado en estos ultimos años como un aparato de entretenimiento. Los UAV se comenzaron a utilizar hace muchos años con fines b\'elicos, y poco a poco su desarrollo ha permitido que se utilicen en \'ambitos muy diferentes, como la rob\'otica, gracias, por ejemplo, al comportamiento aut\'onomo de \'este, trabajando tambi\'en en una navegaci\'on en interiores. 

\hspace{1cm} Durante este proyecto se ha diseñado y programado un algoritmo con la finalidad de que el drone navegue de forma aut\'onoma y no necesite a ninguna persona dici\'endole lo que tiene que hacer en cada momento o controlandole, sino que se pueda confiar en el una vez abandone el punto de origen. Para conseguir esto hay que tener en cuenta los distintos estados en los que se puede encontrar el drone (como puede ser despegando, en periodo de b\'usqueda o aterrizaje), situandose en cada uno de estos dependiendo de lo que este viendo en cada instante. Para el desarrollo de la visi\'on se ha trabajado en la detecci\'on visual objetos, haciendo esto a partir de filtros de color, que nos permite detectar si un objeto es o no el deseado y de esta forma poder dirigirnos a \'el. Cabe destacar que este tiene que ser un objeto bien elegido y que sea dificil de confundir con los dem\'as, para poder asegurarnos de un correcto funcionamiento. Para toda esta programaci\'on nos hemos apoyado en el entorno JdeRobot, utilizando OpenCV para el procesado de im\'agenes.

\hspace{1cm} El algoritmo programado se ha validado experimentalmente, tanto en un drone simulado (en simulador Gazebo), como en un drone real (utilizando el Ardrone2 de parrot). Se ha conseguido un comportamiento satisfactorio funcionando en tiempo real. Todo el software desarrollado y los v\'ideos de las pruebas est\'a accesible p\'ublicamente. 


 

